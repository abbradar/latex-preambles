% polyglossia should go first!
\usepackage{polyglossia} % multi-language support

\usepackage{amsmath} % math symbols, new environments and stuff
\usepackage{unicode-math} % changing math font and unicode symbols
\usepackage[style=english]{csquotes} % fancy quoting
\usepackage{microtype} % better font rendering
\usepackage{hyperref} % refs and URLs
\usepackage{graphicx} % images (and title page)
\usepackage[top=2cm, left=3cm, bottom=2cm, right=2cm]{geometry} % margins in title page
\usepackage{tabu} % tabulars (and title page)
\usepackage[backend=bibtex, style=numeric-comp, sorting=none]{biblatex} % bibliography
\usepackage{placeins} % float barriers
\usepackage{titlesec} % section break hooks
\usepackage{listings} % listings
\usepackage{upquote} % good-looking quotes in source code (used for custom languages)
\usepackage{xcolor} % colors!
\usepackage{enumitem} % unboxed description labels (long ones)
\usepackage{caption} % captions
\usepackage{tocloft} % dots in TOC
\usepackage{indentfirst} % indentation after section
\usepackage{csvsimple} % importing csv files
\usepackage{multirow} % multirows in tables
\usepackage{tikz} % graphics

\setmainlanguage{russian}
\setotherlanguage{english}
\defaultfontfeatures{Mapping=tex-text} % for converting "--" and "---"
\setmainfont{CMU Serif}
\setsansfont{CMU Sans Serif}
\setmonofont{CMU Typewriter Text}
\setmathfont{XITS Math}
\DeclareSymbolFont{letters}{\encodingdefault}{\rmdefault}{m}{it} % for Russian in math
\MakeOuterQuote{"} % enable auto-quotation

% new page and barrier after section, also phantom section after clearpage for
% hyperref to get right page.
% clearpage also outputs all active floats:
\newcommand{\sectionbreak}{\clearpage\phantomsection}
\newcommand{\subsectionbreak}{\FloatBarrier}
\newcommand\numberthis{\addtocounter{equation}{1}\tag{\theequation}}
\renewcommand{\thesection}{\arabic{section}} % no chapters
\numberwithin{equation}{section}

\newcommand{\icode}[1]{\texttt{#1}} % inline code

\lstset{
  numbers=left,
  numberstyle=\scriptsize,
  basicstyle=\ttfamily\scriptsize,
  columns=fullflexible,
  keepspaces, % for spaces in unicode text!
  captionpos=b % Russian standards, again
}

\renewcommand{\figurename}{Рисунок}
\renewcommand{\lstlistingname}{Листинг}

\makeatletter
\define@key{blx@lbx}{fromjapanese}{\blx@defstring{fromjapanese}{#1}}
\define@key{blx@lbx}{langjapanese}{\blx@defstring{langjapanese}{#1}}
\makeatother

% \addglobalbib{doc.bib}
